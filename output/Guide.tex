\documentclass[
  12pt,
  a4paper,
  oneside,
]{article}

% ---------------- パッケージ ----------------
\usepackage{xcolor}
\usepackage{geometry}
\usepackage{amsmath,amssymb}
\usepackage{fontspec}
\usepackage{graphicx}
\usepackage{hyperref}
\usepackage{setspace}
\usepackage{titlesec}
\usepackage{tabularx}
\usepackage{booktabs}
\usepackage{longtable}
\usepackage{array}
\usepackage{ragged2e}
\usepackage{fancyhdr}
\usepackage{xparse}
\usepackage{pgffor}
\usepackage{xstring}
\usepackage{etoolbox}
\usepackage{pgfmath}
\usepackage{microtype}
\usepackage{xeCJK}
\usepackage{xfp}
\usepackage{float}
\usepackage{ltablex}
\usepackage{makecell}
\keepXColumns
\usepackage[most]{tcolorbox}
\tcbuselibrary{listingsutf8}  % UTF-8対応(不要なら削除可)

% ---------------- 書式 ----------------
\geometry{margin=20mm}
\setmainfont{Segoe UI}
\setCJKmainfont{Yu Gothic UI}
\XeTeXlinebreaklocale "ja"
\XeTeXlinebreakskip = 0em plus 0.1em minus 0.01em
\fancyhead[R]{\thepage}
\fancyfoot{}
\sloppy
\setlength{\parindent}{1em}
\setlength{\parskip}{0.5em}
\setlength{\overfullrule}{0pt}

% ---------------- 表 ----------------
\renewcommand{\arraystretch}{1.1}
\setlength{\tabcolsep}{4pt}
\newcolumntype{Y}{>{\RaggedRight\arraybackslash}X}
\setlength{\LTpre}{0pt}
\setlength{\LTpost}{0pt}

% ---------------- セクション ----------------
\newcommand{\secTitleWithLine}[1]{%
  #1\\[-0.8em]
  \noindent{\color{gray!50}\rule{\dimexpr\textwidth-1.5em}{0.4pt}}%
}
\titleformat{\section}
  [hang]
  {\Large\bfseries}
  {\thesection}
  {1em}
  {\secTitleWithLine}
\titleformat{\subsection}{\large\bfseries}{\thesubsection}{1em}{}

% ---------------- 画像マクロ ----------------
\newcommand{\SingleImage}[1]{%
  \par\noindent\centering
  \includegraphics[
    width=0.9\linewidth,
    height=0.75\textheight,
    keepaspectratio
  ]{#1}
  \par\vspace{1em}
}
\newcommand{\InsertImageRow}[1]{%
  \begingroup
  \newcount\imgcount
  \imgcount=0
  \foreach \x in {#1} { \advance\imgcount by 1 }
  \noindent\centering
  \foreach \x [count=\i from 1] in {#1} {%
    \ifnum\imgcount=1
      \includegraphics[
        width=0.95\textwidth,
        height=0.8\textheight,
        keepaspectratio
      ]{\x}%
    \else
      \ifnum\imgcount=2 \def\imgwidth{0.47\textwidth}%
      \else\ifnum\imgcount=3 \def\imgwidth{0.31\textwidth}%
      \else\ifnum\imgcount=4 \def\imgwidth{0.23\textwidth}%
      \else               \def\imgwidth{0.19\textwidth}%
      \fi\fi\fi
      \includegraphics[
        width=\imgwidth,
        height=0.5\textheight,
        keepaspectratio
      ]{\x}%
    \fi
    \ifnum\i<\imgcount\hspace{0.02\linewidth}\fi
  }%
  \par\vspace{1em}
  \endgroup
}

% ------------------ コードブロック風スタイル ------------------
% コマンド例やコードスニペットを見やすく表示するための設定。
% 背景色付きの枠で囲み、等幅フォント(\ttfamily)を使って明確に区別。


\tcbuselibrary{listingsutf8}

% 「コードブロック風」マクロの定義
\newtcolorbox{CodeBlockBox}{
  colback=gray!10,       % 背景色(明るい灰色)
  colframe=gray!60,      % 枠線色(やや濃い灰色)
  boxrule=0.3mm,         % 枠線の太さ
  arc=1mm,               % 角の丸み
  left=1mm, right=1mm,   % 左右の内側余白
  top=0.5mm, bottom=0.5mm, % 上下の内側余白
  fontupper=\ttfamily,   % 中のテキストを等幅フォントに(コードらしさを出す)
  enhanced,              % 見た目の改善を有効にする
  breakable              % 複数行に渡っても枠が壊れないようにする
}

% ---------------- Pandoc マクロ補完 ----------------
\providecommand{\tightlist}{%
  \setlength{\itemsep}{0pt}\setlength{\parskip}{0pt}
}

% ---------------- 本文 ----------------
\begin{document}
\raggedbottom
\begin{center}
  {\huge\bfseries Markdown文書作成ガイドライン(PDF出力対応)} \par
        \vspace{1em}
    {\large XXXXXX} \par
        {\normalsize 2025年4月} \par
    \vspace{0.5em}
  {\color{gray!70}\rule{\textwidth}{0.5pt}}\\[-0.7em]
  {\color{gray!70}\rule{\textwidth}{0.5pt}}
\end{center}
\vspace{2em}

\section{概要}\label{ux6982ux8981}

このガイドラインは、Pandoc と LaTeX テンプレートを用いて Markdown ファイルから PDF を安定して生成するための記述ルールを定めたものです。

\section{一般設定(出力時の基本オプション)}\label{ux4e00ux822cux8a2dux5b9aux51faux529bux6642ux306eux57faux672cux30aaux30d7ux30b7ux30e7ux30f3}

\begin{CodeBlockBox}
\texttt{pandoc Guide.md -o Guide.pdf \\
  --template=template\_custom.latex \\
  --pdf-engine=xelatex \\
  --columns=120}
\end{CodeBlockBox}

\begin{itemize}
\tightlist
\item
  ``--pdf-engine=xelatex'':日本語フォントに対応。
\item
  ``--columns=120'' は自動折り返しや表幅の制御のために必要です。
\end{itemize}

\section{セクション構成と自動番号付け}\label{ux30bbux30afux30b7ux30e7ux30f3ux69cbux6210ux3068ux81eaux52d5ux756aux53f7ux4ed8ux3051}

\# Markdownの\verb|#|,\verb|##|\textbar,\verb|###|\textbar{} などを使えば、自動で LaTeX の \textbackslash section,
\textbackslash subsection, \textbackslash subsubsection に変換され、番号は自動で付与されます。

\begin{CodeBlockBox}
\verb|#| 第1章のタイトル

\verb|##| 1.1 セクションタイトル

\verb|###| 1.1.1 小見出し
\end{CodeBlockBox}

❌ セクション番号を手動で書かないでください(PDFで重複やズレの原因になります)。

\section{表のルール}\label{ux8868ux306eux30ebux30fcux30eb}

\subsection{書き方の基本}\label{ux66f8ux304dux65b9ux306eux57faux672c}

\begin{itemize}
\tightlist
\item
  通常のMarkdown表記でOK。
\item
  ただし、文字数に制限がある。目安は以下の通り
\end{itemize}

\begin{longtable}[]{@{}ll@{}}
\toprule\noalign{}
項目 & 推奨1列あたりの文字数(全角) \\
\midrule\noalign{}
\endhead
\bottomrule\noalign{}
\endlastfoot
2列 & \textasciitilde40文字 \\
3列 & \textasciitilde30文字 \\
4列 & \textasciitilde20文字 \\
\end{longtable}

\subsection{長文セルの改行方法}\label{ux9577ux6587ux30bbux30ebux306eux6539ux884cux65b9ux6cd5}

長文セルでは \textbackslash{} ではなく、LaTeXコマンド \makecell{...} を使う:

\begin{CodeBlockBox}
| 項目 | 説明                                     |
|------|------------------------------------------|
| 例   | \makecell{これは長い文章なので\\改行して見やすくします} |
\end{CodeBlockBox}

↓出力例 \textbar{} 項目 \textbar{} 説明 \textbar{} \textbar------\textbar------------------------------------------\textbar{} \textbar{} 例
\textbar{} \makecell{これは長い文章なので\\改行して見やすくします} \textbar{}

\section{画像挿入のルール}\label{ux753bux50cfux633fux5165ux306eux30ebux30fcux30eb}

画像のレイアウトや大きさは テンプレート側で制御しています。 そのため、画像を挿入したい場合は、以下をマークダウンへ書き込んでください。 \#\#
単体画像

\begin{CodeBlockBox}
\SingleImage{images/xxx.jpg}
\end{CodeBlockBox}

サイズ:縦長の画像は本文高さの40\%になり、横長の画像は本文幅の80\%になる

\subsection{横並び画像}\label{ux6a2aux4e26ux3073ux753bux50cf}

\begin{CodeBlockBox}
\InsertImageRow{{images/img1.jpg}, {images/img2.jpg}, {images/img3.jpg}}
\end{CodeBlockBox}

\begin{itemize}
\tightlist
\item
  一枚だけ指定しても問題なく動作するが、出力される画像は小さくなる。
\item
  最大5毎まで横に並ぶ
\item
  自動で余白や画像幅が調整される
\end{itemize}

\section{改行と段落のルール}\label{ux6539ux884cux3068ux6bb5ux843dux306eux30ebux30fcux30eb}

\begin{longtable}[]{@{}lll@{}}
\toprule\noalign{}
操作 & Markdown記法 & 説明 \\
\midrule\noalign{}
\endhead
\bottomrule\noalign{}
\endlastfoot
段落区切り & 空行を挟む & 通常の段落区切り \\
明示的改行 & ``\textbackslash{}''(非推奨) & 表内だと効かない時がある \\
表内改行 & ``\makecell{a\\b}'' & 表中の複数行対応 \\
\end{longtable}

\section{禁止事項}\label{ux7981ux6b62ux4e8bux9805}

\begin{itemize}
\item
  ❌ 表の中に数式など複雑なLaTeXコマンドを直接書くこと(エラー原因になります)
\item
  ❌ \textbackslash section などの LaTeX 命令をMarkdown中に直接書くこと(Pandocで二重変換)
\item
  ❌ \textbackslash{} で強制改行しすぎる(表や画像と干渉)
\end{itemize}

\section{その他注意事項}\label{ux305dux306eux4ed6ux6ce8ux610fux4e8bux9805}

\begin{itemize}
\item
  改ページは自動に任せましょう(\newpage は使わない)
\item
  ページ内に画像が入りきらない場合、自動的に次ページに送られます
\item
  できる限りテンプレート内で自動制御されるよう調整済みです
\end{itemize}

\section{よくあるエラーと対処}\label{ux3088ux304fux3042ux308bux30a8ux30e9ux30fcux3068ux5bfeux51e6}

\begin{longtable}[]{@{}ll@{}}
\toprule\noalign{}
エラー内容 & 説明 \\
\midrule\noalign{}
\endhead
\bottomrule\noalign{}
\endlastfoot
Missing number, treated as zero & 一列内の文字が多すぎる \\
File ended while scanning use of & ``\{\}''の対応ミス \\
画像が出ない & パスミス or 画像ファイルの存在確認 \\
表がはみ出す & --columns=120 設定 or \makecellで改行 \\
\end{longtable}

\end{document}
