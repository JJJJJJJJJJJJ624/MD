%\usepackage{amsfonts, latexsym}
\usepackage{graphicx}
\usepackage{float}
\usepackage{luatexja}
\usepackage{luatexja-fontspec}
%\setmainfont{Times New Roman}
%\setmainjfont{Yu Mincho}
%\setsansjfont{Yu Mincho}
%\setmonofont{Yu Mincho}


\usepackage{fancyhdr}
\usepackage{lastpage}
\pagestyle{fancy}
\fancyhf{}
\fancyfoot[C]{\thepage{} / \pageref{LastPage}}

\newcommand{\fixedhrule}{
  \noindent\rule{0.8\linewidth}{0.4pt}
}

\usepackage{tcolorbox}
\usepackage{listings}
\tcbuselibrary{listings}

\lstset{
  basicstyle=\ttfamily\small,
  breaklines=true,
  breakatwhitespace=true,
  columns=fullflexible,
  keepspaces=true
}

\tcbset{
  mycode/.style={
    colback=gray!12,
    colframe=gray!60,
    fontupper=\ttfamily\small,
    boxrule=0.4pt,
    arc=2mm,
    left=2mm,right=2mm,top=1mm,bottom=1mm,
    listing only,
    listing options={style=default}
  }
}

% --- ここから 追記 --------------------------------
\usepackage[hang,small,bf]{caption}
\captionsetup[figure]{labelformat=empty}  % 「図 1」のような番号付けをしない
\captionsetup[table]{labelformat=empty}   % テーブル番号も不要なら同様に設定

\usepackage[subrefformat=parens]{subcaption}
\captionsetup{compatibility=false}

% --- ここまで 追記 --------------------------------

\makeatletter
\renewcommand{\maketitle}{
  \begin{center}
    {\LARGE\bfseries \@title \par}
    \vspace{1em}
  \end{center}

  \begin{flushright}  % 右寄せブロック開始
    {\small \@author \par}              % 著者
    \vspace{0.5em}
    {\small \@date \par}                % 日付
    \vspace{0.5em}\rule{\linewidth}{0.4pt} % 薄い横線
  \end{flushright}  

  }
\makeatother

\usepackage{tikz}
\newcommand{\maru}[1]{%
  \tikz[baseline=(char.base)]\node[shape=circle,draw,inner sep=1pt] (char) {#1};%
}
