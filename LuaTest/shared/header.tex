%\usepackage{amsfonts, latexsym}
\usepackage{graphicx}
\usepackage{float}
\usepackage{luatexja}
\usepackage{luatexja-fontspec}
%\setmainfont{Times New Roman}
%\setmainjfont{Yu Mincho}
%\setsansjfont{Yu Mincho}
%\setmonofont{Yu Mincho}


\usepackage{fancyhdr}
\usepackage{lastpage}
\pagestyle{fancy}
\fancyhf{}
\fancyfoot[C]{\thepage{} / \pageref{LastPage}}

\newcommand{\fixedhrule}{
  \noindent\rule{0.8\linewidth}{0.4pt}
}
\usepackage{underscore}

\usepackage{tcolorbox}
\usepackage{listings}
\tcbuselibrary{listings}

\lstset{
  basicstyle=\ttfamily\small,
  breaklines=true,
  breakatwhitespace=true,
  columns=fullflexible,
  keepspaces=true
}

\tcbset{
  mycode/.style={
    colback=gray!12,
    colframe=gray!60,
    fontupper=\ttfamily\small,
    boxrule=0.4pt,
    arc=2mm,
    left=2mm,right=2mm,top=1mm,bottom=1mm,
    listing only,
    listing options={style=default}
  }
}

% --- ここから 追記 --------------------------------
\usepackage[hang,small,bf]{caption}
\captionsetup[figure]{labelformat=empty}  % 「図 1」のような番号付けをしない
\captionsetup[table]{labelformat=empty}   % テーブル番号も不要なら同様に設定

\usepackage[subrefformat=parens]{subcaption}
\captionsetup{compatibility=false}

% --- ここまで 追記 --------------------------------

\usepackage{enumitem}
\usepackage{amssymb}

% 箇条書きのネスト数を 9 段階まで拡張
\setlistdepth{9}
\renewlist{itemize}{itemize}{9}
\setlist[itemize]{label=\textbullet}  % 全階層の基本設定

% 必要に応じて階層ごとの記号を変える
\setlist[itemize,1]{label=\textbullet}
\setlist[itemize,2]{label=--}
\setlist[itemize,3]{label=*}
\setlist[itemize,4]{label=\textopenbullet}
\setlist[itemize,5]{label=\tiny$\blacksquare$}
\setlist[itemize,6]{label=\tiny$\square$}
\setlist[itemize,7]{label=\tiny$\triangleright$}
\setlist[itemize,8]{label=\tiny$\circ$}
\setlist[itemize,9]{label=\tiny$\diamond$}



% --- enumerateのラベル再定義を抑制 ---
\makeatletter
\def\labelenumi{\arabic{enumi}.}
\def\labelenumii{\arabic{enumii}.}
\def\labelenumiii{\arabic{enumiii}.}
\def\labelenumiv{\arabic{enumiv}.}
% pandocの\def再定義を打ち消すため、再度上書き
\AtBeginDocument{
  \def\labelenumi{\arabic{enumi}.}
  \def\labelenumii{\arabic{enumii}.}
  \def\labelenumiii{\arabic{enumiii}.}
  \def\labelenumiv{\arabic{enumiv}.}
}
\makeatother

% enumerate環境を9階層に拡張
\renewlist{enumerate}{enumerate}{9}
\setlist[enumerate]{label=\arabic*.}      % 1.
% 各階層の番号フォーマットを指定
\setlist[enumerate,1]{label=\arabic*.}      % 1.
\setlist[enumerate,2]{label=\alph*)}        % a)
\setlist[enumerate,3]{label=\roman*)}       % i)
\setlist[enumerate,4]{label=\Alph*)}        % A)
\setlist[enumerate,5]{label=\arabic*)}      % 1)
\setlist[enumerate,6]{label=\alph**)}       % a**)
\setlist[enumerate,7]{label=\roman**)}      % i**)
\setlist[enumerate,8]{label=\Alph**)}       % A**)
\setlist[enumerate,9]{label=\arabic**)}     % 1**)